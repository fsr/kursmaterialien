\input{slides_template}	% nothing to do here
\input{c_introduction_info} % TODO modify this if you have not already done so

% meta-information
\usepackage{tikz}
\definecolor{orange}{RGB}{255,127,0}
\newcommand{\topic}{
	Arrays
}

% nothing to do here
\title{\topic}
\supertitle{\course}
\date{\today}

% the actual document
\begin{document}

\maketitle

\begin{frame}{Contents}
	\tableofcontents
\end{frame}

\section{Overview}
\subsection{}
\begin{frame}[fragile]{Hello World explosion}
	Remember the program that printed "Hello World!" by passing 13 chars to \textit{printf}?
	\begin{itemize}
		\item Write a program that prints the following text by passing chars to \textit{printf}.
		\begin{lstlisting}[numbers=none]
Lorem ipsum dolor sit amet, consetetur sadipscing elitr,
sed diam nonumy eirmod tempor invidunt ut labore et
dolore magna aliquyam erat, sed diam voluptua. At vero
eos et accusam et justo duo dolores et ea rebum.
\end{lstlisting}
	\end{itemize}
	\uncover<2->{Just kiddin'}\\
	\uncover<3->{If you thought about iterating through the 200+ variables you defined to solve the problem, let me tell you: There's no way.}
\end{frame}
\begin{frame}{Let's talk about memory}
	\begin{itemize}
		\item Consider a program using 4 characters.
		\item<3->{(Maybe it's a string saying "foo")}\\
		\item<4->{Wouldn't it be nice to iterate through these chars?}
	\end{itemize}
	\begin{tikzpicture}[font=\scriptsize,x=1.4cm]
		\begin{uncoverenv}<2->
			\draw (0,1.6) -- (7,1.6);
			\draw (0,1.6) -- (0,1.9);
			\draw (0,1.9) -- (7,1.9);
			\draw (7,1.6) -- (7,1.9);
			
			\draw[dashed] (.5,1.6) -- (.5,1.9);
			\draw[dashed] (1,1.6) -- (1,1.9);
			\draw[dashed] (1.5,1.6) -- (1.5,1.9);
			\draw[dashed] (2,1.6) -- (2,1.9);
			\draw[dashed] (2.5,1.6) -- (2.5,1.9);
			\draw[dashed] (3,1.6) -- (3,1.9);
			\draw[dashed] (3.5,1.6) -- (3.5,1.9);
			\draw[dashed] (4,1.6) -- (4,1.9);
			\draw[dashed] (4.5,1.6) -- (4.5,1.9);
			\draw[dashed] (5,1.6) -- (5,1.9);
			\draw[dashed] (5.5,1.6) -- (5.5,1.9);
			\draw[dashed] (6,1.6) -- (6,1.9);
			\draw[dashed] (6.5,1.6) -- (6.5,1.9);
		\end{uncoverenv}
		
		\begin{uncoverenv}<5->
			\draw (0,1) -- (7,1);
			\draw (0,1) -- (0,1.3);
			\draw (0,1.3) -- (7,1.3);
			\draw (7,1) -- (7,1.3);		
			
			\draw[dashed] (.5,1) -- (.5,1.3);
			\draw[dashed] (1,1) -- (1,1.3);
			\draw[dashed] (1.5,1) -- (1.5,1.3);
			\draw[dashed] (2,1) -- (2,1.3);
			\draw[dashed] (2.5,1) -- (2.5,1.3);
			\draw[dashed] (3,1) -- (3,1.3);
			\draw[dashed] (3.5,1) -- (3.5,1.3);
			\draw[dashed] (4,1) -- (4,1.3);
			\draw[dashed] (4.5,1) -- (4.5,1.3);
			\draw[dashed] (5,1) -- (5,1.3);
			\draw[dashed] (5.5,1) -- (5.5,1.3);
			\draw[dashed] (6,1) -- (6,1.3);
			\draw[dashed] (6.5,1) -- (6.5,1.3);
		\end{uncoverenv}
		
		\begin{uncoverenv}<2->
			\node[blue, below=.4em, right=0em, ] at (0,1.9) {char};
			\draw[dashed, blue] (0,1.6) -- (.5,1.6);
			\draw[dashed, blue] (.5,1.6) -- (.5,1.9);
			\draw[dashed, blue] (0,1.9) -- (.5,1.9);
			\draw[dashed, blue] (0,1.6) -- (0,1.9);
		
			\node[blue, below=.4em, right=0em, ] at (.5,1.9) {char};
			\draw[dashed, blue] (.5,1.6) -- (1,1.6);
			\draw[dashed, blue] (1,1.6) -- (1,1.9);
			\draw[dashed, blue] (.5,1.9) -- (1,1.9);
			\draw[dashed, blue] (.5,1.6) -- (.5,1.9);
			
			\node[blue, below=.4em, right=0em, ] at (1,1.9) {char};
			\draw[dashed, blue] (1,1.6) -- (1.5,1.6);
			\draw[dashed, blue] (1.5,1.6) -- (1.5,1.9);
			\draw[dashed, blue] (1,1.9) -- (1.5,1.9);
			\draw[dashed, blue] (1,1.6) -- (1,1.9);
		
			\node[blue, below=.4em, right=0em, ] at (1.5,1.9) {char};
			\draw[dashed, blue] (1.5,1.6) -- (2,1.6);
			\draw[dashed, blue] (2,1.6) -- (2,1.9);
			\draw[dashed, blue] (1.5,1.9) -- (2,1.9);
			\draw[dashed, blue] (1.5,1.6) -- (1.5,1.9);
		\end{uncoverenv}
		
		\begin{uncoverenv}<3->
			\draw (.25,2) -- (.25,2.6) node[above]{c1 = 'f';};
			\draw (.75,2) -- (.75,2.2) node[above]{c2 = 'o';};
			\draw (1.25,2) -- (1.25,2.6) node[above]{c3 = 'o';};
			\draw (1.75,2) -- (1.75,2.2) node[above]{c4 = '\textbackslash0';};
		\end{uncoverenv}
		
		\begin{uncoverenv}<7->
			\node[orange, below=.4em, right=0em, ] at (2.8,1.9) {int};
			\draw[dashed, orange] (2,1.6) -- (4,1.6);
			\draw[dashed, orange] (4,1.6) -- (4,1.9);
			\draw[dashed, orange] (2,1.9) -- (4,1.9);
			\draw[dashed, orange] (2,1.6) -- (2,1.9);
			\draw (3,2) -- (3,2.5) node[above]{i1};
			
			\node[orange, below=.4em, right=0em, ] at (4.8,1.9) {int};
			\draw[dashed, orange] (4,1.6) -- (6,1.6);
			\draw[dashed, orange] (6,1.6) -- (6,1.9);
			\draw[dashed, orange] (4,1.9) -- (6,1.9);
			\draw[dashed, orange] (4,1.6) -- (4,1.9);
			\draw (5,2) -- (5,2.5) node[above]{i2};
		\end{uncoverenv}
		
		\begin{uncoverenv}<6->		
			\node[blue, below=.45em, right=0em, ] at (.7,1.3) {char[4]};
			\draw[dashed, blue] (0,1) -- (2,1);
			\draw[dashed, blue] (2,1) -- (2,1.3);
			\draw[dashed, blue] (0,1.3) -- (2,1.3);
			\draw[dashed, blue] (0,1) -- (0,1.3);
			\draw (.25,.9) -- (.25,.5) node[below]{c[0] = 'f';};
			\draw (.75,.9) -- (.75,.1) node[below]{c[1] = 'o';};
			\draw (1.25,.9) -- (1.25,.5) node[below]{c[2] = 'o';};
			\draw (1.75,.9) -- (1.75,.1) node[below]{c[3] = '\textbackslash0';};
		\end{uncoverenv}
		
		\begin{uncoverenv}<7->			
			\node[orange, below=.45em, right=0em, ] at (3.8,1.3) {int[2]};
			\draw[dashed, orange] (2,1) -- (6,1);
			\draw[dashed, orange] (6,1) -- (6,1.3);
			\draw[dashed, orange] (2,1.3) -- (6,1.3);
			\draw[dashed, orange] (2,1) -- (2,1.3);
			\draw (3,.9) -- (3,.4) node[below]{i[0]};
			\draw (5,.9) -- (5,.4) node[below]{i[1]};
		\end{uncoverenv}
	\end{tikzpicture}\\
	\uncover<5->{C offers an opportunity to access variables through an index: Arrays}
\end{frame}

% nothing to do from here on
\end{document}
