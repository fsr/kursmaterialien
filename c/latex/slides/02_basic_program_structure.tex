\input{slides_template}	% nothing to do here
\input{c_introduction_info} % TODO modify this if you have not already done so

% meta-information
\usepackage{ulem}

\newcommand{\topic}{
	Basic programm structure
}

% nothing to do here
\title{\topic}
\supertitle{\course}
\date{\today}

% the actual document
\begin{document}

\maketitle

\begin{frame}{Contents}
	\tableofcontents
\end{frame}

\section{Overwiew}
\subsection{}
\begin{frame}[fragile]{A basic program}
	\begin{columns}[T]
		\column{.6\textwidth}
		\begin{lstlisting}
#include <stdio.h>

int main(int argc, char* argv[]) {

	printf("Hello World!\n");
	/* Print "Hello World!" */

	return 0;
}
\end{lstlisting}
		\column{.4\textwidth}
		
		\ \\$\left. \begin{array}{c}\\\end{array}\right\rbrace $ preprocessor statements
		\ \\\ \\$\left. \begin{array}{c}\\\\\\\\\\\end{array}\right\rbrace $ main function
	\end{columns}
\end{frame}
\section{Preprocessor part}
\subsection{}
\begin{frame}[fragile]{Preprocessor statements}
	\begin{lstlisting}
#include <stdio.h>
\end{lstlisting}
\begin{itemize}
	\item processed before compilation
	\item inclusion of external files or libraries
	\item definition of constants and macros, e.g.
\end{itemize}
\begin{lstlisting}[numbers=none]
#define THE_ANSWER 42
\end{lstlisting}
\end{frame}
\section{Main function}
\subsection{}
\begin{frame}[fragile]{The main function}
	\begin{itemize}
		\item basic function
		\item exists only once per program
		\item called on program start
	\end{itemize}
	\StartLineAt{3}
	\begin{lstlisting}
int main(int argc, char* argv[]) {
\end{lstlisting}
	\begin{itemize}
		\item as a function, \textit{main()} takes parameters
		\item get used to \textit{argc} and \textit{argv}, even if it will be explained later
		\item $\lbrace$ starts the main function scope
	\end{itemize}
\end{frame}
\begin{frame}[fragile]{The main function}
	\StartLineAt{5}
	\begin{lstlisting}
	printf("Hello World!\n");
	/* Print "Hello World!" */
\end{lstlisting}
	\begin{itemize}
		\item all program statements
		\item they are processed from top to bottom
	\end{itemize}
	\StartLineAt{8}
	\begin{lstlisting}
	return 0;
}
\end{lstlisting}
	\begin{itemize}
		\item last statement
		\item ends main function
		\item \textit{0} tells the OS that everything went right
		\item $\rbrace$ ends the main function scope
	\end{itemize}
\end{frame}
\begin{frame}[fragile]{Statements}
	\begin{itemize}
		\item instructions for the computer
		\item seperated by \textit{;} (semicolon)
		\item there is the empty statement:
	\end{itemize}
	\begin{lstlisting}[numbers=none]
	;
\end{lstlisting}
	\begin{itemize}
		\item some lines don't do anything, e.g.
	\end{itemize}
	\begin{lstlisting}[numbers=none]
	/* me does nothing */
	// me neither
\end{lstlisting}
\end{frame}
\begin{frame}[fragile]{Comments}
	\begin{itemize}
		\item are cut out before compilation
		\item do not appear in the final program
	\end{itemize}
	Single line comments:
	\begin{lstlisting}[numbers=none]
	// this is an easter egg
\end{lstlisting}
	Block comments:
	\begin{lstlisting}[numbers=none]
	/* this is a bigger
	easter egg. */
\end{lstlisting}
	Better use of block comments:
	\begin{lstlisting}[numbers=none]
	/*
	 * this is an even bigger
	 * easter egg.
	 */
\end{lstlisting}
\end{frame}
\section{Style}
\subsection{}
\begin{frame}[fragile]{A few words on style}
	\begin{itemize}
		\item
			\only<-2>{there can be multiple statements on one line}
			\only<3->{\sout{there can be multiple statements on one line}}
		\item
			\only<-2>{intendation is not nessessary at all}
			\only<3->{\sout{intendation is not nessessary at all}}
		\item<2-> BUT...
	\end{itemize}
	\ \\
	\begin{uncoverenv}<2->
	\begin{lstlisting}[numbers=none]
#include <stdio.h>
int
main(int argc, char* argv[]){printf("Hello World!\n");
		// Prints
/*"Hello World!"			*/
		return 0;}
\end{lstlisting}
	\end{uncoverenv}
	\begin{itemize}
		\item<3-> put each statement on a single line
		\item<3-> intend every statement in the main function by one tab
		\item<3-> use \textit{/* ... */} rather than \textit{// ...}
	\end{itemize}
\end{frame}
% nothing to do from here on
\end{document}
