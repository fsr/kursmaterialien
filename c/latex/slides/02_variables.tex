%% Nothing to modify here.
%% make sure to include this before anything else

\documentclass[10pt]{beamer}
\usetheme{Szeged}

% packages
\usepackage{color}
\usepackage{listings}

% color definitions
\definecolor{mygreen}{rgb}{0,0.6,0}
\definecolor{mygray}{rgb}{0.5,0.5,0.5}
\definecolor{mymauve}{rgb}{0.58,0,0.82}

% re-format the title frame page
\makeatletter
\def\supertitle#1{\gdef\@supertitle{#1}}%
\setbeamertemplate{title page}
{
  \vbox{}
  \vfill
  \begin{centering}
  \begin{beamercolorbox}[sep=8pt,center]{title}
      \usebeamerfont{supertitle}\@supertitle
   \end{beamercolorbox}
    \begin{beamercolorbox}[sep=8pt,center]{title}
      \usebeamerfont{title}\inserttitle\par%
      \ifx\insertsubtitle\@empty%
      \else%
        \vskip0.25em%
        {\usebeamerfont{subtitle}\usebeamercolor[fg]{subtitle}\insertsubtitle\par}%
      \fi%     
    \end{beamercolorbox}%
    \vskip1em\par
    \begin{beamercolorbox}[sep=8pt,center]{author}
      \usebeamerfont{author}\insertauthor
    \end{beamercolorbox}
    \begin{beamercolorbox}[sep=8pt,center]{institute}
      \usebeamerfont{institute}\insertinstitute
    \end{beamercolorbox}
    \begin{beamercolorbox}[sep=8pt,center]{date}
      \usebeamerfont{date}\insertdate
    \end{beamercolorbox}\vskip0.5em
    {\usebeamercolor[fg]{titlegraphic}\inserttitlegraphic\par}
  \end{centering}
  \vfill
}
\makeatother

% insert frame number
\expandafter\def\expandafter\insertshorttitle\expandafter{%
      \insertshorttitle\hfill%
\insertframenumber\,/\,\inserttotalframenumber}

% preset-listing options
\lstset{
  backgroundcolor=\color{white},   
  % choose the background color; 
  % you must add \usepackage{color} or \usepackage{xcolor}
  basicstyle=\footnotesize,        
  % the size of the fonts that are used for the code
  breakatwhitespace=false,         
  % sets if automatic breaks should only happen at whitespace
  breaklines=true,                 % sets automatic line breaking
  captionpos=b,                    % sets the caption-position to bottom
  commentstyle=\color{mygreen},    % comment style
  % deletekeywords={...},            
  % if you want to delete keywords from the given language
  extendedchars=true,              
  % lets you use non-ASCII characters; 
  % for 8-bits encodings only, does not work with UTF-8
  frame=single,                    % adds a frame around the code
  keepspaces=true,                 
  % keeps spaces in text, 
  % useful for keeping indentation of code 
  % (possibly needs columns=flexible)
  keywordstyle=\color{blue},       % keyword style
  % morekeywords={*,...},            
  % if you want to add more keywords to the set
  numbers=left,                    
  % where to put the line-numbers; possible values are (none, left, right)
  numbersep=5pt,                   
  % how far the line-numbers are from the code
  numberstyle=\tiny\color{mygray}, 
  % the style that is used for the line-numbers
  rulecolor=\color{black},         
  % if not set, the frame-color may be changed on line-breaks 
  % within not-black text (e.g. comments (green here))
  stepnumber=1,                    
  % the step between two line-numbers. 
  % If it's 1, each line will be numbered
  stringstyle=\color{mymauve},     % string literal style
  tabsize=4,                       % sets default tabsize to 4 spaces
  title=\lstname                   
  % show the filename of files included with \lstinputlisting; 
  % also try caption instead of title
}

% macro for code inclusion
\newcommand{\includecode}[2][c]{
	\lstinputlisting[caption=#2, style=custom#1]{#2}
}	% nothing to do here
\usepackage[english]{babel}

\usepackage[utf8]{inputenc}

\newcommand{\course}{
	C introduction
}

\author{
	Richard Mörbitz,
	Manuel Thieme
}

\lstset{
	language = C,
	showspaces = false,
	showtabs = false,
	showstringspaces = false,
	escapechar = @,
	belowskip=-1.5em
} % TODO modify this if you have not already done so

% meta-information
\newcommand{\topic}{
	Variables
}

% nothing to do here
\title{\topic}
\supertitle{\course}
\date{\today}

% the actual document
\begin{document}

\maketitle

\begin{frame}{Contents}
	\tableofcontents
\end{frame}

\section{Overview}
\subsection{}
\begin{frame}{Everything's a number}
	In C, the concept of variables has been adopted from mathematics.
	\begin{itemize}
		\item used to remember values during program execution
		\item stored in the memory
		\item accessed by an identifier
		\item for the computer everything looks like a number (sequence of bits)
		\item but is interpreted as different data types
	\end{itemize}
\end{frame}
\begin{frame}[fragile]{Use}
	Declaration: \textit{$<$type$>$ $<$identifier$>$};
	\begin{lstlisting}[numbers=none,belowskip=-1em]
	int number;
\end{lstlisting}
	Assignment: \textit{$<$identifier$>$ = $<$value$>$};
	\begin{lstlisting}[numbers=none,belowskip=-1em]
	number = 42;
\end{lstlisting}
	Definition (all at once): \textit{$<$type$>$ $<$identifier$>$ = $<$value$>$};
	\begin{lstlisting}[numbers=none]
	int another_number = 23;
\end{lstlisting}
\end{frame}
\begin{frame}[fragile]{Saving lines}
	Multiple declarations
	\begin{lstlisting}[numbers=none,belowskip=-1em]
	int number, another_number;
\end{lstlisting}
	Multiple Definitions
	\begin{lstlisting}[numbers=none, belowskip=0em]
	int number = 42, anothernumber = 23;
\end{lstlisting}
But be careful:
\begin{columns}[c]
	\column{.4\textwidth}
	\begin{lstlisting}[numbers=none]
int a = 23, b = 23;
\end{lstlisting}
	\column{.12\textwidth}
	\centering
	$\neq$
	\column{.3\textwidth}
	\begin{lstlisting}[numbers=none]
int a, b = 23;
\end{lstlisting}
	
\end{columns}
\end{frame}
\section{Data types}
\subsection{}
\begin{frame}{Integer numbers}
	\begin{itemize}
		\item keywords: \textit{int}, \textit{short}, \textit{long}
		\item stored as a binary number with fixed length
		\item can be \textit{signed}(default) or \textit{unsigned} 
		\item range: 0 to $2^{length}-1$ (unsigned) or $-2^{length}$ to $2^{length-1}-1$(signed)
		\item actual size of \textit{int}, \textit{short}, \textit{long} depends on architechture
	\end{itemize}
\end{frame}
\begin{frame}{Floating point numbers}
	\begin{itemize}
		\item keywords: \textit{float}, \textit{double}, \textit{long double}
		\item stored as specified in \textit{IEEE 754 Standard} TL;DR
		\item special values for $\infty$, $-\infty$, NaN
		\item useful for fractions and very large numbers
		\item type a decimal point instead of a comma!
	\end{itemize}
\end{frame}
\begin{frame}[fragile]{Characters}
	\begin{itemize}
		\item keyword: \textit{char}
		\item can be \textit{signed}(default) or \textit{unsigned}
		\item range: 0 to 255 (unsigned) or -128 to 127 (signed)	
		\item size: 8 bit (on almost every architechture)
		\item intended to represent a symbol
		\item stores its \textit{ASCII} code number (e.g. 'A' $\Rightarrow$ 65)
		\item you can define a char either by its ASCII code number or by its symbol:
	\end{itemize}
	\begin{lstlisting}[numbers=none]
char a = 65;
char b = 'A';	/* use single quotation marks */
\end{lstlisting}
\end{frame}
\section{Identifiers}
\subsection{}
\begin{frame}{Valid identifiers}
	\begin{itemize}
		\item consist of English letters (no \textit{ß}, \textit{ä}, \textit{ö}, \textit{ü}), numbers and underscore (\_)
		\item start with a letter or underscore
		\item are case sensitive (\textit{number} differs from \textit{Number})
		\item are unique - must not be redeclared
		\item must not be reserved words (e.g \textit{int}, \textit{return})
	\end{itemize}
\end{frame}
\begin{frame}[fragile]{Speaking identifiers}
	\begin{lstlisting}
/* calculate volume of square pyramid */
int a, b, c;
a = 3;
b = 2;
c = (1 / 3) * a * a * b;
\end{lstlisting}
\centering
$\Downarrow$
	\begin{lstlisting}
/* calculate volume of square pyramid */
int length, height, volume;
length = 3;
height = 2;
volume = (1 / 3) * length * length * height;
\end{lstlisting}
\end{frame}
\begin{frame}{Use speaking identifiers.}
	\LARGE
	\centering
	Please, use speaking identifiers.\footnotemark
	
	\footnotetext[1]{Seriously, use speaking identifiers.}
\end{frame}
\begin{frame}[fragile]{Scope}
	In C programs, each identifier has its own scope where it may be used. In the rest of the program, it is invalid and refering it will cause compilation errors. \\
	The scope starts at the line the variable is declared and ends at the line the block, in which the variable was declared, ends. \\
	Blocks begin with a '\{' and end with a '\}'\\\ \\
	\begin{lstlisting}[numbers=none]
{
	int i = 4;
	{
		i = 3;		/* valid */
		int j = 5;
	}
	j = 2;			/* invalid, j is not in scope */
}
\end{lstlisting}
\end{frame}
\begin{frame}[fragile]{the difference between \textit{same} and \textit{equal}}
	What if?
	\begin{lstlisting}
#include <stdio.h>

int main(char argc, char* argv[]) {
	int i = 3;
	{
		i = 2;
		int i = 4;
		printf("%d", i);	/* prints the value of i */
	}
	printf("%d\n", i);		/* prints the value of i */
}
\end{lstlisting}
	Try it yourself and discuss the output.
\end{frame}
\section{Variable I/O}
\subsection{}
\begin{frame}[fragile]{\textit{printf()} with placeholders}
	The string you pass to \textit{printf} can contain placeholders:
	\begin{lstlisting}[numbers=none]
int a = 3, b = 5;
float c = 7.4;
printf("a: %d\n", a);
printf("b: %d\nc: %f\n", b, c);
\end{lstlisting}
Output:\begin{lstlisting}[numbers=none]
a: 3
b: 5
c: 4.7
\end{lstlisting}
You can insert any amount of placeholders. For each placeholder, you have to pass a value of the corresponding type.
\end{frame}
\begin{frame}{Example placeholders}
	The placeholder determines how the variable is interpreted.
	To avoid compiler warnings, only use the following combinations: \\ \ \\
	\begin{tabular}{|c|c|c|}
		\hline
		\textbf{type} & \textbf{description} & \textbf{type of argument} \\\hline
		\%c & single character & char, int (if $<=$ 255) \\\hline
		\%d & decimal number & char, int \\\hline
		\%u & unsigned decimal number & unsigned char, unsigned int \\\hline
		\%X & hexadecimal number & char, int \\\hline
		\%ld & long decimal number & long \\\hline
		\%f & floating point number & float, double \\\hline
	\end{tabular}
\end{frame}
\begin{frame}[fragile]{Variable input}
	scanf() is another useful function from the standard library.
	\begin{itemize}
		\item Like \textit{printf()}, it is declared in \textit{stdio.h}
		\item Like \textit{printf()}, it has a format string with placeholders
		\item You can use it to read values of primitive datatypes from the command line
	\end{itemize}
	\ \\ \ \\ Example:
	\begin{lstlisting}[numbers=none]
int i;
scanf("%d", &i);	
\end{lstlisting}
	After calling \textit{scanf()}, the program waits for the user to input a value in the command line.
	After pressing the \textit{return} key, that value is stored in \textit{i}.
\end{frame}
\begin{frame}{Note:}
	\begin{itemize}
		\item \textit{scanf()} uses the same placeholders as \textit{printf()}
		\item You must type an \textit{\&} before each variable identifier \\
			(will be explained that later)
		\item If you read a number (using \%d, \%u etc.), interpretation
		\begin{itemize}
			\item starts at first digit
			\item ends before last non digit character
		\end{itemize}
		\item If you use \%c, the first character of the user input is interpreted (this may be a ' ' as well!)
	\end{itemize}
	Never trust the user: they may enter a blank line while you expect a number, which means your input variable is still undefined!
		
\end{frame}
\section{Exercise}
\subsection{}

\begin{frame}[fragile]{Hello again!}
	\begin{itemize}
	
		\item Write a program that prints the String "Hello World!" on the command line, using only placeholders in the format string.
		\item \textbf{Experts:} In leetspeek you say "H3110 W0$|$21d!". Use actual numbers.

	\end{itemize}
\end{frame}
\begin{frame}[fragile]{ASCII code explorer}
	\begin{itemize}
		\item Write a program that asks to input a character and prints its ASCII code number
		\begin{itemize}
			\item<2-> Hint: it is a matter of interpretation
		\end{itemize}

	\end{itemize}	
\end{frame}
% nothing to do from here on
\end{document}
