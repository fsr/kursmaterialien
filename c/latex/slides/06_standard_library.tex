%% Nothing to modify here.
%% make sure to include this before anything else

\documentclass[10pt]{beamer}
\usetheme{Szeged}

% packages
\usepackage{color}
\usepackage{listings}

% color definitions
\definecolor{mygreen}{rgb}{0,0.6,0}
\definecolor{mygray}{rgb}{0.5,0.5,0.5}
\definecolor{mymauve}{rgb}{0.58,0,0.82}

% re-format the title frame page
\makeatletter
\def\supertitle#1{\gdef\@supertitle{#1}}%
\setbeamertemplate{title page}
{
  \vbox{}
  \vfill
  \begin{centering}
  \begin{beamercolorbox}[sep=8pt,center]{title}
      \usebeamerfont{supertitle}\@supertitle
   \end{beamercolorbox}
    \begin{beamercolorbox}[sep=8pt,center]{title}
      \usebeamerfont{title}\inserttitle\par%
      \ifx\insertsubtitle\@empty%
      \else%
        \vskip0.25em%
        {\usebeamerfont{subtitle}\usebeamercolor[fg]{subtitle}\insertsubtitle\par}%
      \fi%     
    \end{beamercolorbox}%
    \vskip1em\par
    \begin{beamercolorbox}[sep=8pt,center]{author}
      \usebeamerfont{author}\insertauthor
    \end{beamercolorbox}
    \begin{beamercolorbox}[sep=8pt,center]{institute}
      \usebeamerfont{institute}\insertinstitute
    \end{beamercolorbox}
    \begin{beamercolorbox}[sep=8pt,center]{date}
      \usebeamerfont{date}\insertdate
    \end{beamercolorbox}\vskip0.5em
    {\usebeamercolor[fg]{titlegraphic}\inserttitlegraphic\par}
  \end{centering}
  \vfill
}
\makeatother

% insert frame number
\expandafter\def\expandafter\insertshorttitle\expandafter{%
      \insertshorttitle\hfill%
\insertframenumber\,/\,\inserttotalframenumber}

% preset-listing options
\lstset{
  backgroundcolor=\color{white},   
  % choose the background color; 
  % you must add \usepackage{color} or \usepackage{xcolor}
  basicstyle=\footnotesize,        
  % the size of the fonts that are used for the code
  breakatwhitespace=false,         
  % sets if automatic breaks should only happen at whitespace
  breaklines=true,                 % sets automatic line breaking
  captionpos=b,                    % sets the caption-position to bottom
  commentstyle=\color{mygreen},    % comment style
  % deletekeywords={...},            
  % if you want to delete keywords from the given language
  extendedchars=true,              
  % lets you use non-ASCII characters; 
  % for 8-bits encodings only, does not work with UTF-8
  frame=single,                    % adds a frame around the code
  keepspaces=true,                 
  % keeps spaces in text, 
  % useful for keeping indentation of code 
  % (possibly needs columns=flexible)
  keywordstyle=\color{blue},       % keyword style
  % morekeywords={*,...},            
  % if you want to add more keywords to the set
  numbers=left,                    
  % where to put the line-numbers; possible values are (none, left, right)
  numbersep=5pt,                   
  % how far the line-numbers are from the code
  numberstyle=\tiny\color{mygray}, 
  % the style that is used for the line-numbers
  rulecolor=\color{black},         
  % if not set, the frame-color may be changed on line-breaks 
  % within not-black text (e.g. comments (green here))
  stepnumber=1,                    
  % the step between two line-numbers. 
  % If it's 1, each line will be numbered
  stringstyle=\color{mymauve},     % string literal style
  tabsize=4,                       % sets default tabsize to 4 spaces
  title=\lstname                   
  % show the filename of files included with \lstinputlisting; 
  % also try caption instead of title
}

% macro for code inclusion
\newcommand{\includecode}[2][c]{
	\lstinputlisting[caption=#2, style=custom#1]{#2}
}	% nothing to do here
\usepackage[english]{babel}

\usepackage[utf8]{inputenc}

\newcommand{\course}{
	C introduction
}

\author{
	Richard Mörbitz,
	Manuel Thieme
}

\lstset{
	language = C,
	showspaces = false,
	showtabs = false,
	showstringspaces = false,
	escapechar = @,
	belowskip=-1.5em
} % TODO modify this if you have not already done so

% meta-information
\newcommand{\topic}{
	The C standard library
}

% nothing to do here
\title{\topic}
\supertitle{\course}
\date{\today}

% the actual document
\begin{document}

\maketitle

\begin{frame}{Contents}
	\tableofcontents
\end{frame}

\section{Introduction}
\subsection{}
\begin{frame}{Don't reinvent the wheel}
	If you split your problem into sub-problems and solve each of them in a seperate function, you're on a good way. \\ \ \\
	However, many of the very basic sub-problems have already been solved ages ago. \\ \ \\
	These solution are provided in \textit{libraries} such as the \textit{glibc} used by the \textit{gcc}. \\ \ \\
	Library implemetations are safer and more efficient than yours will ever be, \textit{plus} you save a lot of time using them.
\end{frame}
\begin{frame}[fragile]{The Hitchhiker's Guide to the standard library}
	The functions are \textit{declared} in a \textit{header file}. \\
	Each header file has a certain name and the file extension \textit{.h}. \\ \ \\
	The \textit{include} preprocessor statement puts them into your program, e.g.
\begin{lstlisting}[numbers=none]
#include <stdio.h>	/* We have done that so many times */
\end{lstlisting} \ \\ \ \\foo
	The actual function implementation is linked dynamically to your program during runtime. Let's not care about that. \\ \ \\
	With less than 30 header files, the C library is rather small. \\
	We will go through the ones that might be the most useful to you.
\end{frame}
\section{Useful headers}
\subsection{}
\begin{frame}[fragile]{assert.h}
	 \begin{itemize}
	 	\item Contains the \textit{assert()} macro, witch evaluates the truth value of an expression
	 	\item If it's true, nothing happens
	 	\item Else the program aborts and an error message is printed
	 \end{itemize} \ \\ \ \\
	 $\rightarrow$ useful to avoid undefined behaviour / worse errors at runtime \\ \ \\
	 We can also use it if we just want to test things:
	 \begin{lstlisting}[numbers=none]
unisgned int input;
printf("Enter a one-digit decimal number:\n");
scanf("%d", &input);
assert(input < 10);
\end{lstlisting}
\end{frame}
\begin{frame}[fragile]{math.h}
	\begin{itemize}
		\item Declares a lot of mathematic functions
		\item Finally you're able to calculate square roots, logarithms, etc.
		\item Most of those functions have \textit{double} arguments and return values
	\end{itemize} \ \\ \ \\
	If you use functions from \textit{math.h}, add the \textit{-lm} as the \textbf{last} option to \textit{gcc} to avoid compiler errors:
	\begin{lstlisting}[numbers=none]
gcc main.c -lm
\end{lstlisting}
\end{frame}
\begin{frame}[fragile]{stdio.h}
	\begin{itemize}
		\item Declares the basic functions to read and write data
		\item You know \textit{printf()} and \textit{scanf()}, but this can do more:
		\item Characters, unprocessed and formatted strings
		\item Command line I/O and file access
		\item Many functions for high-level file management
	\end{itemize} \ \\ \ \\
	As an example, \textit{puts()} can be used instead of \textit{printf()} if you have a basic string without placeholders - '\textit{\textbackslash n}' is added automatically:
	\begin{lstlisting}[numbers=none]
puts("Hello World!");
/* Equivalent to printf("Hello World!\n") */
\end{lstlisting}
\end{frame}
\begin{frame}{stdlib.h}
	This probably is the most powerful header providing various different functionalities. Here is just an excerpt:
	\begin{itemize}
		\item \textit{EXIT\_SUCCESS} and \textit{EXIT\_FAILURE} constants as an alternative to returning \textit{0} or \textit{1} at the end of \textit{main()}
		\item Alternative ways to exit the program
		\item Generation of pseudo-random numbers
		\item Search and sorting function
		\item Dynamic memory management
	\end{itemize}
		\dots and more things you haven't even heard of
\end{frame}
\begin{frame}{string.h}
	\begin{uncoverenv}<2->
		Wait! Strings? \\ \ \\
	\end{uncoverenv}
	\begin{uncoverenv}<3->
		Yes, there are strings in C. \\
		They are just handled differently from what you would expect. \\ \ \\
		\textit{string.h} is crucial if you want to work with C strings seriously. \\
		We will use some of the functions declared there in later lessons.
	\end{uncoverenv}
\end{frame}
\begin{frame}{time.h}
	\begin{itemize}
		\item Data types to store different time formats
		\item Functions to get the calendar and cpu time
		\item Functions to format time values
		\item Functions to measure and calculate time differences
	\end{itemize} \ \\ \ \\
	There's nothing more about \textit{time.h}, but since dealing with time is pretty complicated it's nice that other people already took care of it.
\end{frame}

\section{Man page}
\subsection{}
\begin{frame}[fragile]{Documentation}
	Instead of explaining every single function in detail, it's better to tell you where too look them up quickly. \\
	\textit{Man page} is a Unix tool containing documentation of programs, system calls and libraries - such as the C standard library. \\ \ \\
	To access a certain man page, just type:
	\begin{lstlisting}[numbers=none, basicstyle=\itshape\small]
$ man page
\end{lstlisting} \ \\ \ \\
Example for \textit{printf()}:
	\begin{lstlisting}[numbers=none]
$ man printf
\end{lstlisting}
However, this describes the shell command \textit{printf}.
\end{frame}
\begin{frame}[fragile]{Effective use of \textit{man}}
	Man has many sections, library functions are in \#3. \\
	Write the section number between \textit{man} and the page:
	\begin{lstlisting}[numbers=none]
$ man 3 printf
\end{lstlisting} \ \\ \ \\
	To get all pages \textit{printf} occurs in, use the \textit{-k} option:
	\begin{lstlisting}[numbers=none]
$ man -k printf
\end{lstlisting} \ \\ \ \\
	If you need more information on \textit{man} - it has its own man page:
	\begin{lstlisting}[numbers=none]
$ man man
\end{lstlisting}
\end{frame}
\section{Exercise}
\subsection{}
\begin{frame}{Pretty output}
	To solve the following tasks, you need to use some functions that may be new to you. \\
	No explanation is given on them, use \textit{man} to find out how they work. \\ \ \\
	\begin{itemize}
		\item Write a program that outputs a function table like the following:
		\item In the first row, print all values $x$ from 1 to 10. In the second row, print $1/x$ from 1 to 10.
		\item Play around with the format string of \textit{printf()} to create a nice table-like output.
		\item \textbf{Experts}: Print a multiplication table instead (from $1*1$ to $10*10$)
	\end{itemize}
\end{frame}
\begin{frame}{Aleia iacta est}
	\begin{itemize}
		\item Write a program that simulates a dice using the \textit{rand()} function.
		\begin{itemize}
			\item<2-> Hint: have a look at \textit{srand()} as well.
		\end{itemize}
		\item \textbf{Experts}: Simulate "real" randomness that cannot be recreated.
		\begin{itemize}
			\item<3-> Hint: choose the \textit{seed} for \textit{srand()} dynamically.
			\item<4-> Hint: what value is different every\textit{time} you start the program?
		\end{itemize}
	\end{itemize}
\end{frame}
\begin{frame}{Math exam}
	\begin{itemize}
		\item Write a program that asks the user to solve some simple mathematic tasks.
		\item If the user answers wrong, the program should abort with an error.
		\begin{itemize}
			\item<2-> Remember the \textit{assert()} macro?
		\end{itemize}
		\item \textbf{Experts}: At the end, print the time it took the user to answer all the questions.
	\end{itemize}
\end{frame}
% nothing to do from here on
\end{document}
