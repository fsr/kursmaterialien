\input{slides_template}	% nothing to do here
\input{c_introduction_info} % TODO modify this if you have not already done so

% meta-information
\newcommand{\topic}{
	User input
}

% nothing to do here
\title{\topic}
\supertitle{\course}
\date{\today}

% the actual document
\begin{document}

\maketitle

\begin{frame}{Contents}
	\tableofcontents
\end{frame}

\section{Reading user input}
\subsection{}
\begin{frame}[fragile]{scanf()}
	scanf() is another useful function from the standard library.
	\begin{itemize}
		\item Like \textit{printf()}, it is declared in \textit{stdio.h}
		\item Like \textit{printf()}, it has a format string with placeholders
		\item You can use it to read values of primitive datatypes from the command line
	\end{itemize}
	\ \\ \ \\ Example:
	\begin{lstlisting}[numbers=none]
int i;
scanf("%d", &i);	
\end{lstlisting}
	After calling \textit{scanf()}, the program waits for the user to input a value in the command line.
	After pressing \textit{return} that value is stored into \textit{i}.
\end{frame}
\begin{frame}{Note:}
	\begin{itemize}
		\item \textit{scanf()} uses the same placeholders as \textit{printf()}
		\item You must type an \textit{\&} before each variable you pass \\ (we will explain that later)
		\item If you read a number (using \%d, \%u etc.), interpretation
		\begin{itemize}
			\item starts at first digit
			\item ends before last non digit character
		\end{itemize}
		\item If you use \%c, the first character of the user input is interpreted (this may be a ' ' as well!)
	\end{itemize}
	Never trust the user: they may enter a blank line while you expect a number, which means your input variable is still undefined!
		
\end{frame}
\section{Exercise}
\subsection{}
\begin{frame}{The stage is yours}
	\begin{itemize}
		\item Write a Program that asks the user for two decimal numbers and prints their sum, difference, product, integer quotient and remainder.
		\item Write a Program that asks the user for 5 characters and prints them in reversed order.
		\item Write a Program that calculates the area of a circle.\\(Let's assume $\pi$ == 3.14)
	\end{itemize}
\end{frame}

% nothing to do from here on
\end{document}
